\documentclass{article}
\usepackage{graphicx} % Required for inserting images
\usepackage[spanish]{babel}

\title{Progressive Web Apps}
\author{Erin Abraham Hernandez Miranda}
\date{10 de Enero 2024}


\begin{document}

\maketitle

\newpage

\section*{Conceptos basicos}
\begin{itemize}
    \item\textbf{Aplicación Web Progresiva:}
    Una aplicación web que utiliza tecnologías modernas para proporcionar una experiencia similar a la de aplicaciones nativas.
    \item\textbf{Service Worker:}
    Un script JavaScript que se ejecuta en segundo plano, permitiendo funcionalidades como el manejo de eventos de red y el trabajo offline.
    \item\textbf{Manifesto Web}
    Un archivo JSON que describe la aplicación, proporcionando información sobre su nombre, iconos, colores y configuración.
    \item\textbf{Responsive Web Design}
    Diseño web que se adapta automáticamente a diferentes tamaños de pantalla y dispositivos.
    \item\textbf{HTTPS}
    Protocolo de transferencia seguro que cifra la comunicación entre el navegador y el servidor, es esencial para las PWA.
    \item\textbf{Offline-First:}
    Enfoque de desarrollo que prioriza el funcionamiento de la aplicación incluso en ausencia de conexión a Internet.
    \item\textbf{Navegador Service Worker Compatible:}
    Navegadores web que admiten la ejecución de service workers, como Google Chrome, Mozilla Firefox, Microsoft Edge y Safari.
    \item\textbf{IndexedDB:}
    Base de datos en el navegador que permite el almacenamiento de datos para su uso offline.
    \item\textbf{Cache API:}
    Interfaz que permite el almacenamiento en caché de recursos web para un acceso rápido, especialmente útil en situaciones offline.
\end{itemize}

\newpage

\section*{Requisitos de Instalación para PWA:}
\begin{itemize}
    \item\textbf{Servidor HTTP:}
     Para simular un entorno de servidor. Puedes usar Node.js con herramientas como Express.
    \item\textbf{Certificado SSL:}
     Para habilitar HTTPS y asegurar la comunicación. Puedes obtener certificados de Let's Encrypt.
    \item\textbf{Manifesto Web:}
    Archivo JSON que define la apariencia y comportamiento de la PWA.
    \item\textbf{Service Worker:}
    Un script JavaScript que gestiona las solicitudes de red y permite funcionalidades offline.
\end{itemize}

\section*{Sistemas Operativos:}
\begin{itemize}
    \item\textbf{Windows:}
    Todas las herramientas y tecnologías mencionadas son compatibles.
    \item\textbf{MacOS:}
    Las herramientas principales, como Visual Studio Code y los navegadores populares, son compatibles.
    \item\textbf{Linux:}
    La mayoría de las herramientas y tecnologías son compatibles. Puedes usar editores de código como Visual Studio Code.
\end{itemize}

\section*{Mejor Entorno}
\begin{itemize}
    \item\textbf{Vs code}
    Ampliamente utilizado, ligero, con extensiones para PWA, soporte para múltiples lenguajes y sistemas operativos.
\end{itemize}

\end{document}