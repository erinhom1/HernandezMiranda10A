\documentclass{article}
\usepackage{graphicx} % Required for inserting images
\usepackage[spanish]{babel}

\title{PWA Ventajas y desventajas}
\author{Erin Abraham Hernandez Miranda}
\date{10 de Enero 2024}


\begin{document}

\maketitle

\newpage

\section*{Que es una PWA (Progressive Web App)}
Una Aplicación Web Progresiva (PWA) es una aplicación web que utiliza tecnologías modernas para proporcionar una experiencia de usuario similar a la de una aplicación nativa. 

\section*{Que es una Web App}
es un software interactivo que se accede y utiliza a través de un navegador web. A diferencia de las aplicaciones tradicionales que debes instalar en tu dispositivo, las aplicaciones web se ejecutan en servidores y se entregan al usuario a través de la web. 

\section*{Que es una Web de servicios}
Estas páginas web ofrecen servicios en línea, como la programación de citas, reservas de servicios, consultas en línea, y más. Simplifican la interacción entre proveedores de servicios y clientes


\newpage

\section*{Ventajas}
\begin{itemize}
    \item \textbf{Accesibilidad multiplataforma:} Funciona en diversos dispositivos y plataformas, como computadoras, tablets y teléfonos móviles, eliminando la necesidad de desarrollar aplicaciones específicas para cada sistema operativo.
    \item \textbf{Instalación opcional:} Los usuarios pueden acceder a la PWA a través de un navegador web sin necesidad de descargarla desde una tienda de aplicaciones. Sin embargo, la instalación opcional permite a los usuarios agregar un acceso directo en sus dispositivos.
    \item \textbf{Experiencia offline mejorada:} Las PWAs pueden funcionar sin conexión, permitiendo a los usuarios acceder a ciertas funciones incluso cuando no tienen conexión a Internet.
    \item \textbf{Actualizaciones automáticas:} Las actualizaciones se realizan de manera automática, lo que garantiza que los usuarios siempre tengan acceso a la versión más reciente de la aplicación.
    \item \textbf{Rápido rendimiento:} Utilizan tecnologías como Service Workers para mejorar la velocidad de carga y ofrecer una experiencia de usuario más fluida.
\end{itemize}

\section*{Desventajas}
\begin{itemize}
    \item \textbf{Limitaciones de hardware:} Al depender del navegador, algunas funcionalidades avanzadas pueden no ser accesibles debido a restricciones de hardware o políticas de seguridad del navegador.
    \item \textbf{Menos visibilidad en tiendas de aplicaciones:} Al no estar en una tienda de aplicaciones, las PWAs pueden tener una menor visibilidad en comparación con las aplicaciones nativas que se promocionan en plataformas de distribución.
    \item \textbf{Compatibilidad del navegador:} Aunque la compatibilidad con PWAs ha mejorado, algunas características pueden no ser totalmente compatibles con todos los navegadores, lo que puede limitar su alcance.
    \item \textbf{Almacenamiento limitado:} Las PWAs tienen restricciones en el almacenamiento local, lo que puede limitar la cantidad de datos que se pueden almacenar en el dispositivo del usuario.
\end{itemize}
 

\end{document}
